\section{Conclusion}

In conclusion, our exploration into the ball pen production system has made strides toward a more intelligent and efficient manufacturing approach. Drawing inspiration from key literature, especially \cite{Liu2022}, we crafted a prototype that blends Kafka and MQTT technologies. This strategic combination ensures our system functions smoothly with small sensors while offering scalability to handle increased data demands.
The pivotal scalability test gauged the system's capability to manage heightened workloads with the addition of more production elements. Encouragingly, our prototype demonstrated the ability to maintain data transmission times below 1000 milliseconds, even with the integration of twenty sensors. This success not only affirms the system's adaptability for accommodating additional production cells but also positions us for future developments.
Nevertheless, it's crucial to acknowledge that our current prototype, while showing promise in scalability, remains in its early stages. Further refinements are necessary, including simulating additional system components, scrutinizing attributes related to availability and deployability, and conducting more in-depth testing under realistic workloads.
Our endeavor transcends academic pursuits, aiming to contribute to the evolution of manufacturing processes by infusing intelligence and adaptability. This project serves as a foundational step for ongoing exploration and advancements in manufacturing. Envisioning a future where manufacturing processes are not just intelligent but continually improving, we anticipate the unfolding paradigm shift in adaptable and efficient production systems aligned with Industry 4.0.
In summary, the presented prototype and its evaluation substantiate the viability of the proposed system architecture, representing a key advancement toward intelligent and scalable manufacturing processes. This work provides insights into the future landscape of manufacturing, fostering a pathway for the integration of smart and dynamic production systems that align with the requirements of Industry 4.0.
\cite{Liu2022}